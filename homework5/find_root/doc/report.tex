%! Author = jiang-ziyang
%! Date = 22-7-2

% Preamble
\documentclass{ctexart}

% Packages
\usepackage{graphicx}
\usepackage{amsmath}
\usepackage{amsfonts}
\usepackage{amssymb}
\usepackage{listings}
\usepackage{xcolor}
\usepackage[backref]{hyperref}
\usepackage{cite}
\lstset{
    basicstyle          =   \sffamily,          % 基本代码风格
    keywordstyle        =   \bfseries,          % 关键字风格
    commentstyle        =   \rmfamily\itshape,  % 注释的风格,斜体
    stringstyle         =   \ttfamily,  % 字符串风格
    flexiblecolumns,
    breaklines          =   true, %对过长的代码自动换行
    numbers             =   left,   % 行号的位置在左边
    showspaces          =   false,  % 是否显示空格,显示了有点乱,所以不现实了
    numberstyle         =   \zihao{-5}\ttfamily,    % 行号的样式,小五号,tt等宽字体
    showstringspaces    =   false,
    captionpos          =   t,      % 这段代码的名字所呈现的位置,t指的是top上面
    frame               =   lrtb,   % 显示边框
}

\title{roots.c功能分析}
\author{杨泽加 \\ 3190104662}

% Document
\begin{document}
    \maketitle

    \begin{tabular}{p\columnwidth}
        这是一个求根公式的测试用例,通过二分法获得5的求根迭代结果。\\
        它的输出如下:
    \end{tabular}\\
    \begin{lstlisting}
using brent method
 iter [    lower,     upper]      root        err  err(est)
    1 [1.0000000, 5.0000000] 1.0000000 -1.2360680 4.0000000
    2 [1.0000000, 3.0000000] 3.0000000 +0.7639320 2.0000000
    3 [2.0000000, 3.0000000] 2.0000000 -0.2360680 1.0000000
    4 [2.2000000, 3.0000000] 2.2000000 -0.0360680 0.8000000
    5 [2.2000000, 2.2366300] 2.2366300 +0.0005621 0.0366300
Converged:
    6 [2.2360634, 2.2366300] 2.2360634 -0.0000046 0.0005666
    \end{lstlisting}
    \begin{tabular}{p\columnwidth}
        如上所示,其中 iter 是迭代次数, [lower, upper]是迭代区间,
        root 是跟, err 是root值与真值的误差, err(est)是区间长度。\\
        由代码
    \end{tabular}\\
    \begin{lstlisting}
        int iter = 0, max_iter = 100;
        \\ details
        status = gsl_root_test_interval (x_lo, x_hi, 0, 0.001);
    \end{lstlisting}
    \begin{tabular}{p\columnwidth}
        知,当误差绝对值小于0.001(或迭代次数超过100次)时结束迭代
    \end{tabular}
\end{document}