%! Author = jiang-ziyang
%! Date = 22-6-28
\documentclass{ctexart}

\usepackage{graphicx}
\usepackage{amsmath}
\usepackage{ctex}
\usepackage{amsfonts}
\usepackage{amssymb}
\usepackage{listings}
\usepackage{xcolor}
\lstset{
    basicstyle          =   \sffamily,          % 基本代码风格
    keywordstyle        =   \bfseries,          % 关键字风格
    commentstyle        =   \rmfamily\itshape,  % 注释的风格,斜体
    stringstyle         =   \ttfamily,  % 字符串风格
    flexiblecolumns,
    breaklines          =   true, %对过长的代码自动换行
    numbers             =   left,   % 行号的位置在左边
    showspaces          =   false,  % 是否显示空格,显示了有点乱,所以不现实了
    numberstyle         =   \zihao{-5}\ttfamily,    % 行号的样式,小五号,tt等宽字体
    showstringspaces    =   false,
    captionpos          =   t,      % 这段代码的名字所呈现的位置,t指的是top上面
    frame               =   lrtb,   % 显示边框
}


\title{作业二:自动移动文件}


\author{杨泽加 \\ 统计学 3190104662}

\begin{document}

    \maketitle


    \section{问题介绍:}
    \begin{tabular}{p{\columnwidth}}
        我在ubuntu系统下搭载了一个hexo博客的本地库。但出于写作习惯,我经常在windows系统
        下的obsidian软件里写博客。如何更“自动化”地同步我的文件就成为了一件需要解决的问题。     \\
        我将我obsidian的所有文章都通过git上传到一个private repo中,然后通过git的
        sparsecheckout功能将存储博客文件的两个文件夹 ./blog\_updating/ 和
        ./blog\_complete/ clone 到我的hexo 博客的./source/ 文件夹中。 \\
        接下来,我需要将这两个文件夹内(如果git repo有更改的话)的文件移动到我的./source/
        文件目录下,然后删去两个文件夹以避免干扰到博客渲染工作。有时候,我还需要知道更改了哪些文件。     \\
        这些工作在命令行上都能很好地完成,但都需要多条指令。所以我便产生了将它们写成脚本的想法。
    \end{tabular}\\


    \section{代码如下:}
    \begin{lstlisting}
#! /bin/bash

# pull the blog from remote repo
# and get the return string
information=`git pull origin main`

# if there is nothing changed
# we needn't to operate

if [ "$information" = "已经是最新的。" ]; then
    echo "There isn't uploaded blog in the remote repo"
    exit 0
else
    # else we need to move the changed file into ./ directory
    # the blog file dire is ./blog_updating and ./blog_completing
    for direc in $(ls); do
        if [ -d $direc ]; then
            echo "in directory $direc, file:"
            for file in ./$direc/*; do
                echo -n "$file; "
            done
            mv ./$direc/* ./
            rm -rf ./$direc
            echo "moved successfully"
        fi
    done

    echo "blog source uploaded successfully"
fi

exit 0
    \end{lstlisting}


    \section{测试结果:}
    \begin{tabular}{p{\columnwidth}}
        当有文件更新时,shell script会自动从git repo pull更新过的博文,并将它们转移到
        我的./source/ 目录下。并告诉我转移的文件名。\\
        示例如下:
        若git有更新文件则转移文件并返回
    \end{tabular}\\
    \begin{lstlisting}
来自 github.com:JiangZiyang0120/obsidian
 * branch            main       -> FETCH_HEAD
in directory blog_complete, file:
./blog_complete/linus如何通过SMTP服务器发送QQ邮件.md; moved successfully
in directory blog_updating, file:
./blog_updating/机器学习.md; ./blog_updating/黎明静悄悄.md; ./blog_updating/也许_无问西东.md; moved successfully
blog source uploaded successfully
    \end{lstlisting}
    \begin{tabular}{p{\columnwidth}}
        否则返回(事实上再运行一次即可):
    \end{tabular}\\
    \begin{lstlisting}
来自 github.com:JiangZiyang0120/obsidian
 * branch            main       -> FETCH_HEAD
There isn't uploaded blog in the remote repo
    \end{lstlisting}

    \section{总结:}
    \begin{tabular}{p{\columnwidth}}
        通过上述脚本的练习,我学到了script变量的使用,if、for循环的使用方式等。
        同时,它也给了我一个方便的工具管理我的hexo博客博文。\\
    \end{tabular}
\end{document}
