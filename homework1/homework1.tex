\documentclass{ctexart}

\usepackage{graphicx}
\usepackage{amsmath}
\usepackage{ctex}
\usepackage{amsfonts}
\usepackage{amssymb}


\title{作业一: 洛必达法则的叙述与证明}


\author{杨泽加 \\ 统计学 3190104662}

\begin{document}

    \maketitle

    \begin{tabular}{p{0.9\columnwidth}}
        这是一个来自微积分学领域的问题,常用于计算具有不定型积分的极限的方法。
        该法则以法国数学家纪尧姆.德.洛必达的名字命名,
        但实际上是由瑞士数学家约翰.伯努利发现。
    \end{tabular}


    \section{问题描述}
    \begin{tabular}{p{0.9\columnwidth}}
        令$c\in \bar{\mathbb{R}}$,两函数$f(x),g(x)$在以$x=c$为端点的开区间可微,存在 \\
        \begin{equation}
            \lim_{x\rightarrow c}\frac{f'(x)}{g'(x)}\in\bar{\mathbb{R}},g'(x)\neq 0
            \label{declaration1}
        \end{equation}
        若$\lim_{x\rightarrow c}f(x) = \lim_{x\rightarrow c}g(x) = 0$或
        $\lim_{x\rightarrow c}|f(x)| = \lim_{x\rightarrow c}|g(x)| = \infty$
        其中一者成立,则有欲求的极限
        \begin{equation}
            \lim_{x\rightarrow c}\frac{f(x)}{g(x)} =
            \lim_{x\rightarrow c}\frac{f'(x)}{g'(x)}
            \label{declaration2}
        \end{equation}
    \end{tabular}


    \section{证明}
    \begin{tabular}{p{0.9\columnwidth}}
        令
        \begin{equation}
            l = \lim_{x\rightarrow a^+}\frac{f'(x)}{g'(x)}
            \label{definition1}
        \end{equation}
        我们只对$l$有限的情况证明,当$l = +\infty$或$-\infty$时,证明是类似的。 \\
        先证明$\lim_{x\rightarrow\infty}f(x) = \lim_{x\rightarrow\infty}g(x) = 0$
        条件的式子:
        由定义知:对于任意$\varepsilon>0$,存在$\delta > 0$,当$x\in(a,a+\delta)$时,有:
        \begin{equation}
            l-\varepsilon < \frac{f'(x)}{g'(x)} < l + \varepsilon
            \label{equation1}
        \end{equation}
        所以对于$(x,x_0)\in(a,a+\delta)$,由Cauchy中值定理,必然有$\xi\in(x,x_0)$,使得:
        \begin{equation}
            l-\varepsilon < \frac{f(x) - f(x_0)}{g(x) - g(x_0)} =
            \frac{f'(\xi)}{g'(\xi)} < l + \varepsilon
            \label{equation2}
        \end{equation}
        由恒等变形,我们有:
        \begin{equation}
            \frac{f(x)-f(x_0)}{g(x)-g(x_0)} =
            \frac{\frac{f(x)}{g(x) - \frac{f(x_0)}{g(x)}}}
            {1 - \frac{g(x_0)}{g(x)}}
            \label{equation3}
        \end{equation}
        令 $x+0\rightarrow a$,我们有:$1-\frac{g(x_0)}{g(x)}\rightarrow1$。
        所以我们有:
        \begin{equation}
            \overline{\lim_{x\rightarrow a^+}}\frac{f(x)}{g(x)}\leqslant l+\varepsilon
            \label{equation4}
        \end{equation}
        由$\varepsilon$的任意性,我们有
        \begin{equation}
            \overline{\lim_{x\rightarrow a^+}}\frac{f(x)}{g(x)}\leqslant l
            \label{equation5}
        \end{equation}
        同理可得
        \begin{equation}
            \underline{\lim}_{x\rightarrow a^+}\frac{f(x)}{g(x)}\geqslant l+\varepsilon
            \label{equation6}
        \end{equation}
        结合式\eqref{definition1}有$\lim_{x\rightarrow a^+}\frac{f(x)}{g(x)} = l$。证毕。 \\
        接下来证明当$\lim_{x\rightarrow\infty}\frac{f(x)}{g(x)} = l$的情况:
        存在恒等变形:
        \begin{equation}
            \begin{aligned}
                \frac{f(x)}{g(x)} &= \frac{f(x) - f(x_0)}{g(x)} + \frac{f(x_0)}{g(x)}\\
                &= \frac{g(x) - g(x_0)}{g(x)}\cdot\frac{f(x) - f(x_0)}{g(x) - g(x_0)}
                + \frac{f(x_0)}{g(x)}\\
                &= \left[ 1 - \frac{g(x_0)}{g(x)} \right]\frac{f(x) - f(x_0)}{g(x) - g(x_0)}
                + \frac{f(x_0)}{g(x)}
            \end{aligned}
            \label{equation7}
        \end{equation}
    \end{tabular}
    \newpage
    \begin{tabular}{p{0.9\columnwidth}}
        于是,我们有:
        \begin{equation}
            \begin{aligned}
                \left| \frac{f(x)}{g(x)} - l \right| &=
                \left| \left( 1 - \frac{g(x_0)}{g(x)} \right)
                \frac{f(x) - f(x_0)}{g(x) - g(x_0)} + \frac{f(x_0)}{g(x)} - l \right|\\
                &\leqslant \left| 1 - \frac{g(x_0)}{g(x)} \right|\cdot
                \left| \frac{f(x) - f(x_0)}{g(x) - g(x_0)} - l \right|
                + \left| \frac{f(x_0) - lg(x_0)}{g(x)} \right|
            \end{aligned}
            \label{equation8}
        \end{equation}
        由于:
        \begin{equation}
            \left| \frac{f(x) - f(x_0)}{g(x) - g(x_0)} - l \right|
            = \left| \frac{f'(\xi)}{g'(\xi)} - l \right| < \varepsilon
            \label{equation9}
        \end{equation}
        又由条件:$\lim_{x\rightarrow a^+}g(x) = \infty$,所以当$x$足够接近$a$时,
        \begin{equation}
            \begin{aligned}
                &\left| 1 - \frac{g(x_0)}{g(x)} \right| < C\\
                &\lim_{x\rightarrow a^+}\left|
                \frac{f(x) - f(x_0)}{g(x) - g(x_0)} - l \right| = 0\\
                &\lim_{x\rightarrow a^+}\left| \frac{f(x_0) - g(x_0)}{g(x)} \right| = 0
            \end{aligned}
            \label{equation10}
        \end{equation}
        故:
        \begin{equation}
            \lim_{x\rightarrow a^+}\frac{f(x)}{g(x)} = l =
            \lim_{x\rightarrow a^+}\frac{f'(x)}{g'(x)}
            \label{equation11}
        \end{equation}
        Q.E.D
    \end{tabular}
\end{document}
