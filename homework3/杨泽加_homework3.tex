%! Author = jiang-ziyang
%! Date = 22-6-28

% Preamble
\documentclass{ctexart}

% Packages
\usepackage{graphicx}
\usepackage{amsmath}
\usepackage{amsfonts}
\usepackage{amssymb}
\usepackage{listings}
\usepackage{xcolor}
\usepackage[backref]{hyperref}
\usepackage{cite}
\lstset{
    basicstyle          =   \sffamily,          % 基本代码风格
    keywordstyle        =   \bfseries,          % 关键字风格
    commentstyle        =   \rmfamily\itshape,  % 注释的风格,斜体
    stringstyle         =   \ttfamily,  % 字符串风格
    flexiblecolumns,
    breaklines          =   true, %对过长的代码自动换行
    numbers             =   left,   % 行号的位置在左边
    showspaces          =   false,  % 是否显示空格,显示了有点乱,所以不现实了
    numberstyle         =   \zihao{-5}\ttfamily,    % 行号的样式,小五号,tt等宽字体
    showstringspaces    =   false,
    captionpos          =   t,      % 这段代码的名字所呈现的位置,t指的是top上面
    frame               =   lrtb,   % 显示边框
}



\title{Linux工作环境的介绍}

\author{杨泽加 \\ 统计学 3190104662}


% Document
\begin{document}
    \bibliographystyle{IEEEtran}

    \maketitle
    \section{我的Linux发行版本名称及版本号}\label{S1}
    \begin{tabular}{p\columnwidth}
        通过shell命令 'uname \-a' 获取到我的发行版本名称为\textbf{ubuntu},
        发行版本号为 5.11.0
    \end{tabular}\\
    \begin{lstlisting}
$ uname -a
Linux ubuntu 5.11.0-46-generic #51~20.04.1-Ubuntu SMP Fri Jan 7 06:51:40 UTC 2022 x86_64 x86_64 x86_64 GNU/Linux
    \end{lstlisting}
    \section{我的Linux的配置工作}\label{S2}
    \begin{tabular}{p\columnwidth}
        我为我的linux装配了基本的编译器,如python、GCC、clang等,用于编译我的代码。\\
        于此同时,我在linux上部署了一个hexo博客的本地库,需要用到npm包管理器和hexo渲染器。\\
        当然,我在写代码时也会用到许多库。得益于linux较好的包管理模式,我很少需要更改系统PATH变量\\
        这些包数量较多,难以一一枚举,有openCV\cite{openCV.org}、Eigen3等。\\
        同样的,我也安装了texlive环境以便我方便地编辑和管理内容。\cite{patashnik1984bibtex}
    \end{tabular}\\
    \begin{lstlisting}
~$ python -V
Python 2.7.18
~$ gcc -V
gcc version 9.4.0 (Ubuntu 9.4.0-1ubuntu1~20.04.1)
~$ clang -v
clang version 10.0.0-4ubuntu1
Target: x86_64-pc-linux-gnu
~$ npm -v
6.14.17
~$ hexo -v
hexo-cli: 4.3.0
os: linux 5.11.0-46-generic Ubuntu 20.04.4 LTS (Focal Fossa)
    \end{lstlisting}
    \begin{tabular}{p\columnwidth}
        我的linux系统上装配的软件也以IDE为主。有vim和emacs。但我还是习惯使用
        jetbrains系列的的pycharm、clion和 Rider。其中Rider是虚幻引擎
        Unreal Engine的IDE。同时,我的linux中还编译了QT、UE4
        用于处理偶尔需要的简单动画。\\
        我还安装了typora等markdown文件编辑器。
    \end{tabular}
    \section{对于下一步工作的规划}\label{S3}
    \subsection{下半年使用linux的场景}\label{S3.1}
    \begin{tabular}{p\columnwidth}
        出于日常工作的学习需求,我需要练习我的代码和
        算法能力。我目前较为熟练的计算机语言为C++和python;我均在linux系统内部署了
        较为舒适的环境和IDE。未来有其它需求包也很容易配置。\\
        另外,考虑到我有学习计算机网络的需求,我未来可能会搭建一个树莓派的linux服务器。
        因此仍需学习一些服务器管理工具的使用。\cite{RaspberryPie}\\
        另外,我在该linux系统下部署的hexo博客本地文件也需要探索更多的自动化工作流。
    \end{tabular}
    \subsection{我的工作环境是否符合未来需求}\label{S3.2}
    \begin{tabular}{p\columnwidth}
        目前我的工作环境配置较为齐全和方便。拥有丰富的编译、计算机动画、AI方面的软件和包。
        同时,得益于git工具和github的帮助,我能够轻松地在linux虚拟机和windows10系统中
        协调我的工作。我认为自己仍需学习linux操作系统的功能,并练习诸如LaTeX、UE4等软件。
        但我的linux系统配置已经足以满足我目前所能想到的所有需求了。\\
        需要作出的改变是我需要在我的linux系统中配置科学上网工具。这样就不用切回windows
        系统查阅网页了。也能够更方便我使用git工具。
    \end{tabular}
    \section{如何保证我的工作系统中的代码、文献和工作结果的安全}\label{S4}
    \begin{tabular}{p\columnwidth}
        首先,我拥有一个以上的linux虚拟机。对于一些软件,我会选择先将它部署在我的测试
        虚拟机中,如果没有安全问题再下载/更新至我的主机中。对于软件下载,尽可能
        以编译的方式安装。同时注意软件安装目录的选择,确保自己能够安全地卸载和再部署。\\
        其次,对重要的文件进行备份。我会每隔两到三个月将自己的vm文件备份到我的移动硬盘里。
        许多文件也会通过git上传到github repo里。以避免损失文件的后果。\\
        配置防火墙,防火墙中关键要素有:源IP、目的IP、协议、源端口、目的端口、网卡,
        通过对这几个要素的配置,能有效控制来自外部的攻击及内部的资料泄漏。\\
        尽管我目前没有远程登录ssh的需求,但对此,可以通过修改ssh端口、ssh key认证、禁止root
        用户登录或ssh白名单等方式来保障自己的文件安全和系统不被入侵。\cite{OnlineSafe}
    \end{tabular}
    \bibliography{reference}
\end{document}
